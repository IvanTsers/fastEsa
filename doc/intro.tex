The search for homologous regions between two DNA sequences, a query
and a subject, is a fundamental problem in bioinformatics. Many
programs for sequence comparison involve a step that identifies exact
matches, which serve as seeds for finding an alignment or estimating a
distance between a query and a subject.

An enhanced suffix array (ESA) is a data structure often used in
bioinformatic tools to enable fast exact matching. An ESA consists of
three components:

\begin{enumerate}
    \itemsep0em \item a suffix array (SA): an alphabetically ordered
    array of all suffixes of the subject sequence;
    
    \item a longest common prefix array (LCP): for each position in
    the SA, the LCP stores the length of the longest common
    prefix between the current and previous suffix, enabling fast
    search for common sequence regions;

    \item a child array (CLD): holds information about the topology of
    an implied suffix tree corresponding to the SA. This information
    is used for constant-time navigation between LCP intervals.
\end{enumerate}

The ESA is usually built from the subject sequence, and query
sequences are matched against it. The search process traverses the LCP
array to determine the longest prefix of the query that is also
present in the subject, thus enabling identification of maximal exact
matches.

While exact matching based on ESAs is fast for individual queries,
performance degrades when both query and subject are large genomic
sequences. The large depth of the SA (due to the long subject) and the
long query lead to many LCP lookups during traversal. Since each query
match begins at the root of the implicit LCP-interval tree, the total
time spent on matching becomes significant when many queries are
processed.

However, during repeated searches, shallow intervals in the LCP (those
near the root) are always re-visited, in other words, a repeated work
is being done. This repeatedness presents an opportunity for
optimization. Programs such as \texttt{andi}~\cite{hau15:andi} and
\texttt{phylonium}~\cite{kloe19:phy} use such techniques as bucket
tables (to cache common prefixes) and exploit CPU cache locality to
minimize repeated work and reduce the number of LCP accesses through
skipping the shallow parts of the ESA~\cite{kloe20:thesis}. These
optimizations substantially improve runtime for large-scale exact
matching across many genomic sequences.

The package \texttt{fastEsa} is a \texttt{Go} interface for the
optimized ESA-based exact matching implemented in \texttt{phylonium},
which is a \texttt{C++} program.
